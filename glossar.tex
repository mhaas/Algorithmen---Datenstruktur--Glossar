%% LyX 1.6.5 created this file.  For more info, see http://www.lyx.org/.
\documentclass[english]{scrreprt}
%\documentclass[english]{article}
\usepackage[T1]{fontenc}
\usepackage[latin9]{inputenc}
\setcounter{secnumdepth}{3}
\setcounter{tocdepth}{3}
\usepackage{amsmath}
\usepackage{babel}

\begin{document}

\section*{Begriffe}


\subsection*{Funktionen}
\begin{description}
\item [{monoton~steigend~(fallend)}] Sei f(x) eine Funktion in Abh�ngigkeit von $x$, dann ist die Funktion monoton steigen (fallend), wenn f�r steigendes $x$ die Werte nie echt kleiner (gr��er) werden.
\item [{streng~monoton~steigend~(fallend)}] Sei f(x) eine Funktion in Abh�ngigkeit von $x$, dann ist die Funktion monoton steigen (fallend), wenn f�r steigendes $x$ die Werte immer echt gr��er (kleiner) werden.
\end{description}

\section*{Rechenregeln}


\subsection*{Logarithmus}
\begin{description}
\item [{Logarithmus}] ist die Umkehrfunktion der Exponentialfunktion: 
\item [{$b^{\log_{b}(x)}=x$}] und $\log_{b}(b^{x})=x$

Der Logarithmus beantwortet Fragen wie: wenn ich mit einer Zahl $x$ starte, und dann in jedem Schritt $x=x/y$ setze, wie oft kann das gemacht werden, bis die resultierende Zahl $\leq z$ wird? Das kann so dargestellt werden: $x/(y^i) = z$. Dies ist nat�rlich gleich $z * y^i = x \leftrightarrow y^i = x/z$. Wir fragen nach der Zahl $i$, also m�ssen wir auf beiden Seiten logarithmieren (zum Beispiel zur Basis $2$, dargestellt durch $\log$): $i \log y = \log(x/z) \leftrightarrow i = \frac{\log x - \log z}{\log y}$. In Algorithmen und Datenstrukturen fragen wir besonders h�ufig danach, wie oft man eine Zahl $n$ durch $2$ teilen kann, bis sie $1$ wird. Also: $n/2^i = 1? \leftrightarrow \log n = i$. 
\item [{Produkte}] $\log_{a}(x\times y)=\log_{a}(x)+\log_{a}(y)$
\item [{Quotienten}] $\log_{a}(\frac{x}{y})=\log_{a}(x)-\log_{a}(y)$
\item [{Summen~und~Differenzen}] $\log_{a}(x+y)=\log_{a}(x)+\log_{a}(1+\frac{y}{x})$
\item [{Potenzen}] $log_{a}(x^{r})=r\times log_{a}(x)$
\item [{Basisumrechnung}] von Basis a zu Basis b: $\log_{b}(r)=\frac{\log_{a}(r)}{\log_{a}(b)}$
\end{description}

\section*{Formeln}


\subsection*{Landau-Symbole}

Landau-Symbole verwenden wir, um das asymptotische Verhalten einer
Laufzeitfunktion $f$ zu beschreiben.

F�r $x\rightarrow\infty$:
\begin{description}
\item [{$f\in O(g)$}:] $\exists c>0\exists x_{0}\forall x>x_{0}:|f(x)|\leq c\times|g(x)|$
$\rightarrow$ f w�chst h�chstens so schnell wie g
\item [{$f\in\Omega(g)$}:] $\exists c>0\exists x_{0}\forall x>x_{0}:|f(x)|\ge c\times|g(x)|$
$\rightarrow$ f w�chst mindestens so schnell wie g
\item [{$f\in\Theta(g)$}:] $\exists c_{0}>0\exists c_{1}>0\exists x_{0}\forall x>x_{0}:|c\times|g(x)|\leq|f(x)|\leq c_{1}\times|g(x)|$
$\rightarrow$ f w�chst genauso schnell wie g
\end{description}

\subsubsection*{Achtung: das Gleichheitszeichen}

Oft wird das Gleichheitszeichen verwendet, um auszudr�cken, dass eine
Funktion einer Klasse angeh�rt. So wird auch in der Vorlesung beispielsweise$f(x)=O(x^{2})$
geschrieben, wenn es um eine Funktion quadratischer Laufzeit geht.
Tats�chlich ist das Gleichheitszeichen ein �berladener Operator und
bezeichnet hier nicht die Gleichheit der beiden Ausdr�cke, sondern
wird anstelle des $\in$-Symbols verwendet. $f(x)=O(x^{2})$ hei�t
also nur, dass $f(x)$ in der Menge der Funktionen enthalten ist,
die durch $O(x^{2})$ beschrieben werden. Daher d�rfen diese Gleichheitszeichen auch nur von links nach rechts gelesen werden.
$O(f(n)) = g(n)$ macht im Allgemeinen keinen Sinn! Au�erdem sollten Sie sich diese Schreibweise wenn m�glich nicht angew�hnen, sie wird nur leider in den g�ngigen Artikel und Textb�chern verwendet, so dass Sie sie kennen sollten.


\subsection*{Master-Theorem}

Wenn eine rekursive Gleichung der Laufzeitfunktion in Form $T(n)=\begin{cases}
a & \mbox{f�r n=1}\\
c\times n+d\times T(\frac{n}{b}) & \mbox{f�r n>1}\end{cases}$vorliegt, kann man mit Hilfe des Master-Theorems die Laufzeit bestimmen.

$T(n)=\begin{cases}
\Theta(n) & \mbox{wenn d<b}\\
\Theta(n\times\log(n)) & \mbox{wenn d=b}\\
O(n^{\log_{b}(d)}) & \mbox{wenn d>b}\end{cases}$

Die hier gezeigte Version des Master-Theorems ist beschr�nkt auf Algorithmen
mit linearen Nebenkosten (der Teil mit $c\times n$). In der Literatur
finden sich m�chtigere Varianten.

\subsection*{Beweis durch Induktion}

Mit einem Beweis durch Induktion kann eine Aussage f�r alle
nat�rlichen Zahlen bewiesen werden.

In einem ersten Schritt wird der triviale Fall $n=1$
(oder auch $n=0$) bewiesen: dies ist unser Induktionsanfang.
Wir nehmen an, dass die zu beweisende Aussage f�r beliebige 
Zahlen $n$ gilt (Induktionsannahme).
F�r den Beweis ben�tigen wir den Induktionsschritt: 
wir zeigen (durch Umformung),
dass die Aussage auch f�r $n+1$ gilt.
Ausgehend von unserem Basisfall $n=1$ wissen wir nun,
dass f�r alle folgenden Schritte die Aussage g�ltig ist.
Von $n=1$ kommen wir auf $n=2$ und wissen, dass die Aussage immer
noch g�ltig ist. Von $n=2$ auf $n=3$ etc.\\
\\
Ein einfaches Beispiel:\\
\\
Die zu beweisende Aussage (Multiplikation von n mit 3):\\
$\sum\limits_{1}^n{3}=3\times n$ \\ 
Induktionsanfang: $n=1$:\\
$\sum\limits_{1}^1{3}=3\times 1$ \\
Induktionsschritt mit $n+1$: \\
$\sum\limits_{1}^{n+1}{3}=(\sum\limits_{1}^n{3})+3=(3\times n) + 3 = 3\times (n+1)$\\
Der Trick in der Umformung ist, $\sum\limits_{1}^n{3}$ im zweiten Schritt
durch $3 \times n$ zu ersetzen. Das d�rfen wir, da unsere Induktionsannahme
eben $\sum\limits_{1}^n{3}=3\times n$ lautet und wir annehmen, dass
diese Aussage bereits bewiesen ist.\\

Im �brigen funktioniert diese Beweismethode nicht nur f�r die nat�rlichen
Zahlen, sondern f�r jede fundierte Menge ($\rightarrow$ strukturelle Induktion). 
Es darf also keine unendlich
absteigenden Ketten in der nat�rlichen Ordnung geben, da es sonst
keinen Basisfall ($n=1$) gibt! W�hrend f�r ganze negative Zahlen
ein Induktionsbeweis mit $n-1$ gef�hrt werden kann, funktioniert
es mit der Menge der reellen Zahlen nicht mehr.




\end{document}
