%% LyX 1.6.5 created this file.  For more info, see http://www.lyx.org/.
\documentclass[english]{scrreprt}
\usepackage[T1]{fontenc}
\usepackage[latin9]{inputenc}
\setcounter{secnumdepth}{3}
\setcounter{tocdepth}{3}
\usepackage{amsmath}
\usepackage{babel}

\begin{document}

\section*{Begriffe}


\subsection*{Funktionen}
\begin{description}
\item [{monoton~steigend~(fallend)}] Die Werte, die eine Funktion annimmt,
werden gr��er (kleiner) bzw bleiben konstant.
\item [{streng~monoton~steigend~(fallend)}] Die Werte, die eine Funktion
annimmt, werden gr��er (kleiner).
\end{description}

\section*{Rechenregeln}


\subsection*{Logarithmus}
\begin{description}
\item [{Logarithmus}] ist die Umkehrfunktion der Exponentialfunktion: 
\item [{$b^{\log_{b}(x)}=x$}] und $\log_{b}(b^{x})=x$
\item [{Produkte}] $\log_{a}(x\times y)=\log_{a}(x)+\log_{a}(y)$
\item [{Quotienten}] $\log_{a}(\frac{x}{y})=\log_{a}(x)-\log_{a}(y)$
\item [{Summen~und~Differenzen}] $\log_{a}(x+y)=\log_{a}(x)+\log_{a}(1+\frac{y}{x})$
\item [{Potenzen}] $log_{a}(x^{r})=r\times log_{a}(x)$
\item [{Basisumrechnung}] von Basis a zu Basis b: $\log_{b}(r)=\frac{\log_{a}(r)}{\log_{a}(b)}$
\end{description}

\section*{Formeln}


\subsection*{Landau-Symbole}

Landau-Symbole verwenden wir, um das asymptotische Verhalten einer
Laufzeitfunktion $f$ zu beschreiben.

F�r $x\rightarrow\infty$
\begin{description}
\item [{$f\in O(g)$}] $\exists c>0\exists x_{0}\forall x>x_{0}:|f(x)|\leq c\times|g(x)|$
$\rightarrow$ f w�chst h�chstens so schnell wie g
\item [{$f\in\Omega(g)$}] $\exists c>0\exists x_{0}\forall x>x_{0}:|f(x)|\ge c\times|g(x)|$
$\rightarrow$ f w�chst mindestens so schnell wie g
\item [{$f\in\Theta(g)$}] $\exists c_{0}>0\exists c_{1}>0\exists x_{0}\forall x>x_{0}:|c\times|g(x)|\leq|f(x)|\leq c_{1}\times|g(x)|$
$\rightarrow$ f w�chst genauso schnell wie g
\end{description}

\subsubsection*{Achtung: das Gleichheitszeichen}

Oft wird das Gleichheitszeichen verwendet, um auszudr�cken, dass eine
Funktion einer Klasse angeh�rt. So wird auch in der Vorlesung beispielsweise$f(x)=O(x^{2})$
geschrieben, wenn es um eine Funktion quadratischer Laufzeit geht.
Tats�chlich ist das Gleichheitszeichen ein �berladener Operator und
bezeichnet hier nicht die Gleichheit der beiden Ausdr�cke, sondern
wird anstelle des $\in$-Symbols verwendet. $f(x)=O(x^{2})$ hei�t
also nur, dass $f(x)$ in der Menge der Funktionen enthalten ist,
die durch $O(x^{2})$ beschrieben werden.


\subsection*{Master-Theorem}

Wenn eine rekursive Gleichung der Laufzeitfunktion in Form $T(n)=\begin{cases}
a & \mbox{f�r n=1}\\
c\times n+d\times T(\frac{n}{b}) & \mbox{f�r n>1}\end{cases}$vorliegt, kann man mit Hilfe des Master-Theorems die Laufzeit bestimmen.

$T(n)=\begin{cases}
\Theta(n) & \mbox{wenn d<b}\\
\Theta(n\times\log(n)) & \mbox{wenn d=b}\\
O(n^{\log_{b}(d)}) & \mbox{wenn d>b}\end{cases}$

Die hier gezeigte Version des Master-Theorems ist beschr�nkt auf Algorithmen
mit linearen Nebenkosten (der Teil mit $c\times n$). In der Literatur
finden sich m�chtigere Varianten.
\end{document}
